%#!simpdftex latex relativita.tex
\chapter{Note storiche}
\minitoc Come nasce la relativit\`a ristretta, altrimenti detta
speciale? Nasce dal tentativo, riuscito, di conciliare meccanica
newtoniana ed elettromagnetismo. Einstein stesso scrive a proposito:
\begin{quote}
  \enf{La teoria della relativit\`a nasce necessariamente, per la
    presenza di serie e profonde contraddizioni, dalle quali sembrava
    non ci fosse uscita. La forza della nuova teoria sta nella
    consistenza e semplicit\`a con cui risolve queste difficolt\`a,
    usando poche, ma convincenti, assunzioni.}
\end{quote}
Ma dove queste teorie, elettromagnetismo e meccanica, erano in
disaccordo? 
%Newton sosteneva l'esistenza di un tempo assoluto,
%omogeneo (un tempo uguale per tutti gli osservatori, e indipendente
%dal sistema di riferimento); affermava altres\`i che lo spazio era
%assoluto, omogeneo, ed isotropo. Il sistema di riferimento ideale era
%quello delle stelle, chiamato sistema di riferimento delle stelle
%fisse, che per le conoscenze di allora erano, appunto, considerate
%fisse.  \newline 
Sostanzialmente sembrava che non si trovasse un gruppo di sistemi di
riferimento in cui le leggi dell'elettromagnetismo avessero la stessa
forma, ovvero per avere una descrizione dei fenomeni fisica, sembrava
fosse necessario modificare le leggi dell'elettromagnetismo a seconda
del sistema di riferimento in cui ci si trovava.

Si ipotizz\`o dunque che fossero le leggi per passare da un sistema di
riferimento all'altro ad essere sbagliate, ma a questo punto si rendeva
necessario cambiare le leggi della meccanica da un sistema di
riferimento all'altro.

Prima di poter procedere \`e tuttavia necessario dare
una definizione precisa di sistema di riferimento, in maniera da
evitare errori o imprecisioni:
\begin{definizione}
  Un sistema di riferimento \`e l'insieme di un sistema di coordinate
  e di orologi, sicronizzati tra loro, in grado d'associare ad ogni
  evento un punto dello spazio e un istante di tempo. Un sistema di
  riferimento in cui ogni corpo non soggetto a forze \`e in quiete o
  in moto rettilineo, \`e detto sistema di riferimento inerziale.
\end{definizione}
La definizione precisa di evento verr\`a data pi\`u avanti, per ora
possiamo prendere la nozione euristica che abbiamo di esso. Nei sistemi
di riferimento in moto a velocit\`a costante rispetto a quello delle
stelle fisse, le leggi della meccanica assumevano la stessa forma
tramite le trasformazioni (\ref{eq:trasformazioni}). Quindi la terra
poteva considerarsi un ottimo laboratorio, in quanto
l'accelerazione\index{accelerazione!terrestre} che essa aveva, rispetto
a tale sistema, poteva dirsi trascurabile. Questo era vero se, come
detto, per collegare due sistemi di riferimento si utilizzavano le
(\ref{eq:trasformazioni}). Pi\`u
precisamente, detti $S$ ed $S'$ i due sistemi inerziali, $S'$ in moto
con velocit\`a $\mathsf{v}$ rispetto ad $S$ (velocit\`a costante in
direzione e modulo), si aveva che, se
\begin{displaymath}
  t,\mathbf{x} \mbox{ coordinate in S} \qquad t',\mathbf{x}' \mbox{
    coordinate in S' }
\end{displaymath}
allora
\begin{equation}
  \left\{\begin{array}{l}
      \mathbf{x}'=\mathbf{x}+\mathbf{\mathsf{v}}t\qquad
      (\dagger)\\
      t'=t\end{array}\right. \label{eq:trasformazioni}
\end{equation}
 Si aveva poi l'\enf{invarianza degli intervalli temporali}, ossia:
\begin{equation}
  \Delta t=\Delta t'\label{eq:invtempo}
\end{equation}
e degli \enf{intervalli spaziali}
\begin{equation}
  l = |\Delta\mathbf{x}| =
  |\Delta\mathbf{x}'|=l'\label{eq:invspazio}
\end{equation}  
Vi era poi il:
\begin{teorema}[di addizione delle velocit\`a]
  \index{teorema!di addizione delle velocit\`a}
  \begin{displaymath}
    \mathbf{v}'=\frac{\de}{\de t}\mathbf{x}'=\frac{\de}{\de
      t}\mathbf{x}+\mathbf{\mathsf{v}}=\mathbf{v}+\mathbf{\mathsf{v}}
  \end{displaymath}
\end{teorema}
Tale teorema si ottiene derivando ($\dagger$), ed esso asserisce che
non esiste una velocit\`a assoluta.  
%
\newline 
%
Come anticipato prima, tuttavia, con queste trasformazioni le leggi
dell'elettromagnetismo non potevano essere valide in tutti i sistemi di
riferimento. Vediamo come.
\section{L'etere e l'elettromagnetismo di Maxwell}
Prendiamo l'equazione delle onde elettromagnetiche:
\begin{displaymath}
  \left(\frac{1}{c^2}\cdot\frac{\partial^2}{\partial
      t^2}-\mathbf{\nabla}^2\right)\mathbf{E}(\mathbf{x},t)=0 \qquad
  \mbox{(al posto di $\mathbf{E}$ si pu\`o avere $\mathbf{B}$)};
\end{displaymath}
nell'800 ritenevano che $c$ fosse la velocit\`a della luce nell'etere
luminifero (che porta la luce). Perch\'e si parlava di etere? 
%
%\newline
%
Come prima accennato, le leggi dell'elettromagnetismo non avevano la
stessa forma in tutti i sistemi di riferimento, e quindi l'elegante
forma proposta da Maxwell era corretta solo in un particolare di questi
sistemi. Si ipotizz\`o questo sistema uguale al sistema di riferimento
delle stelle fisse, e lo si ribattezz\`o etere, il sistema in cui le
leggi di Maxwell erano vere senza bisogno di modifiche.
%Si era
%scoperto che le leggi dell'elettromagnetismo, se valide in un
%determinato sistema di riferimento inerziale $S$, non lo erano pi\`u
%in $S'$, sistema in moto a velocit\`a costante rispetto a $S$. 
Vedere che in altri sistemi di riferimento queste modifiche erano
necessarie, non \`e di difficile constatazione: si prenda infatti
un'onda elettromagnetica (che ha la stessa velocit\`a di propagazione
della luce) nel sistema $S$. Essa ha velocit\`a $c$, poich\'e per
l'elettromagnetismo la velocit\`a di propagazione di tale onda \`e
\begin{equation}
  c=\frac{1}{\sqrt{\mu_0\varepsilon_0}};
\end{equation}
se $S'$ si muove con velocit\`a $\rem{v}$ in direzione e verso
dell'onda elettromagnetica, esso rileva la luce avere velocit\`a
$c'=c-\rem{v}$, per il teorema appena visto. Ma allora non sarebbe pi\`u
valida la relazione (derivata dalle normali equazioni di Maxwell)
\begin{displaymath}
  c'=\frac{1}{\sqrt{\mu_0\varepsilon_0}}
\end{displaymath}
essendo $\mu_0$ e $\varepsilon_0$ costanti e \(c'\) la velocit\`a della
luce per l'osservatore di \(S'\). Se ne deduce che una, ed
una sola, delle seguenti ipotesi dev'essere vera:
\begin{description}
\item [Prima ipotesi] Le leggi della meccanica sono covarianti per
  cambiamento di sistema di riferimento inerziale, ma non lo sono le
  leggi dell'elettromagnetismo; le trasformazioni
  (\ref{eq:trasformazioni}) sono giuste.
\item [Seconda ipotesi] Le leggi della meccanica e
  dell'elettromagnetismo sono covarianti per cambiamento di sistemi di
  riferimento inerziale, le trasformazioni (\ref{eq:trasformazioni})
  sono giuste, ma il modo in cui sono scritte le leggi
  dell'elettromagnetismo \`e sbagliato.
\item [Terza ipotesi] Le leggi della meccanica e
  dell'elettromagnetismo sono covarianti per cambiamento di sistemi di
  riferimento inerziale, il modo in cui sono scritte le leggi
  dell'elettromagnetismo \`e giusto (si lascia dunque spazio ad
	    un'eventuale riscrittura delle leggi della meccanica), ma le
	    trasformazioni
  (\ref{eq:trasformazioni}) sono sbagliate.
\end{description}
L'ipotesi corretta la avanz\`o Einstein, e verr\`a affrontata dopo. 
%Se
%l'ipotesi giusta fosse la seconda, esisterebbe un sistema privilegiato,
%quello in cui le onde si propagano con velocit\`a $c$; in tale sistema,
%detto etere, sarebbero valide le equazioni di Maxwell; si pensava tale
%sistema fosse il sistema di riferimento delle stelle fisse.

Nelle tre ipotesi abbiamo fatto uso del termine ``covarianti'':
cosa significa? Una legge fisica \`e covariante quando ha la stessa
forma per tutti i sistemi di riferimento inerziali; ad esempio nel
sistema di riferimento $S$ la seconda legge
di Newton si scrive 
\begin{displaymath}
 \bef{F} = m\bef{a}; 
\end{displaymath}
cambiando sistema di riferimento, sar\`a necessario usare delle
trasformazioni di coordinate per poter scrivere la legge nel nuovo
sistema; ebbene, usando le trasformazioni di Galileo e le leggi della
meccanica conosciute, nel nuovo sistema di riferimento la
``trasformazione'' della seconda legge di Newton \`e
\begin{displaymath}
  \bef{F'} = m'\bef{a'},
\end{displaymath}
dove \(m = m'\): dunque la legge della meccanica ha la stessa forma in
entrambi i sistemi di riferimento, e si pu\`o parlare di covarianza.

%Come sopra detto, $c$ era la velocit\`a della luce in tale sistema. La
%domanda che a quel tempo ci si poneva era: 

Assumendo dunque che l'esistenza dell'etere, era possibile trovare il
moto della terra rispetto a tale sistema?  Vediamo ora due esperimenti
che rispondono a tale domanda in maniera inconciliabile.
\subsection{Esperimento sull'aberrazione}
\index{aberrazione} \setlength{\unitlength}{0.7mm}
\begin{wrapfigure}[17]{r}{4cm}
  \begin{picture}(50,75)
    \put(5,5){\line(1,0){40}}\multiput(25,5)(0,10){7}{\line(0,1){5}}
    \put(22,70){\Huge$\star$} \put(20,5){\line(0,1){25}}
    \put(20,30){\line(1,0){10}}\put(30,30){\line(0,-1){25}}
    \put(28,5){\dashbox{2}(10,25){}}\put(30,17){\huge$\rightarrow$}
    \put(32,21){v}
  \end{picture}
  \caption{Luce stellare incidente sul telescopio}
  \label{fig:telescopio}

\end{wrapfigure}
\setlength{\unitlength}{1mm} Consideriamo il telescopio in figura
\ref{fig:telescopio}. Ci proponiamo di andare a verificare l'angolo di
inclinazione che esso deve avere per poter osservare la luce
proveniente da un corpo celeste. Se la terra si muovesse con una certa
velocit\`a rispetto a tale stella, le cose andrebbero come in figura,
e il telescopio, senza alcun angolo di inclinazione, sarebbe
inservibile. Infatti da quando il raggio della stella entra nel
telescopio, a quando raggiunge l'ordinata dell'obbiettivo
(consideriamo la stella, qualsiasi essa sia, solidale all'etere), il
telescopio si sposta e la luce non colpisce l'obbiettivo. Quindi \`e
necessario inclinare il telescopio. Chiamiamo l'angolo di inclinazione
$\vartheta$ (\`e l'angolo formato dalla terra e dal lato pi\`u corto
dell'obbiettivo). Per valutare approssimativamente la velocit\`a della
terra rispetto al sistema delle stelle fisse, consideriamo il sole
come una stella fissa; la distanza terra sole vale
$R_{\scriptscriptstyle TS}\:=\:150\cdot10^9 \:m$. Il periodo di
rivoluzione attorno al sole vale
$T\:=\:365\:gg\:\Longrightarrow\:\mathsf{v}\:=\:3\cdot10^4\:\rem{m}/\rem{s}$.
Se il telescopio \`e lungo $l$ (farsi un disegno per meglio capire),
il tempo di percorrenza della luce \`e $t=l \cos \vartheta /c$. La
distanza percorsa dal telescopio in $t$ \`e $x=\mathsf{v}t$.  L'angolo
$\vartheta$ risulta dunque:
\begin{displaymath}
  \tan\vartheta= \frac{\mathsf{v} t}{l \cos \vartheta} =
  \frac{\mathsf{v}t}{ct}\approx10^{-4}
\end{displaymath}
Come si vede quest'angolo \`e molto piccolo, ma le misure sono in
accordo con la teoria, cio\`e \`e effettivamente necessario inclinare
i telescopi di $\vartheta$ per poter osservare i raggi
celesti.\footnote{Tuttavia quest'esperimento non fu considerato
  decisivo, poich\'e esso utilizzava le apprissimazioni di ottica
  geometrica dell'elettromagnetismo.}
\subsection{L'esperimento di Michelson e Morley}
\index{Michelson-Morley, esperimento di}L'esperimento di Michelson e
Morley, effettuato per la prima volta nel 1881, era stato ideato per
trovare la velocit\`a della luce rispetto all'etere.  L'esperimento
\`e schematizzato in figura \vref{fig:mm}.
\setlength{\unitlength}{0.5mm}
\begin{figure}[htb]
  \begin{center}
    \begin{picture}(110,100)
      \put(5,50){\vector(1,0){65}}\put(60,40){\line(1,1){20}}
      \put(70,50){\vector(0,1){30}} \put(14,37){\tiny Specchio
        semi-rifrangente}
      \put(60,80){\line(1,0){20}}\multiput(60,80)(5,0){5}{\line(1,3){3}}
      \put(70,80){\vector(0,-1){30}}\put(70,50){\vector(1,0){30}}
      \put(100,40){\line(0,1){20}}
      \multiput(100,40)(0,5){5}{\line(3,1){8}}
      \put(70,50){\vector(0,-1){30}} \put(60,63){\vector(0,-1){13}}
      \put(55,64){$L$} \put(60,67){\vector(0,1){13}}
      \put(83,40){\vector(-1,0){13}}\put(87,40){\vector(1,0){13}}
      \put(85,62){\tiny Specchio} \put(84,37){$L$} \put(83,80){\tiny
        Specchio} \put(5,55){\tiny Sorgente}
      \put(66.5,14.5){\Huge$\blacksquare$} \put(62,8){\tiny
        Rilevatore} \put(3,45.8){\Huge$\blacktriangleright$}
    \end{picture}
    \caption{Schema dell'esperimento di Michelson e Morley, visto nel
      sistema di riferimento solidale alla terra} \label{fig:mm}
  \end{center}
\end{figure}
Spieghiamo cosa succede. La terra si muove con velocit\`a $\mathsf{v}$
rispetto all'etere (secondo le convinzioni dell'epoca). Dunque i raggi
che colpivano lo specchio semirifrangente, e si propagavano in
direzione verticale, facevano un percorso, nel sistema di riferimento
etere, come quello in figura \vref{fig:mim1} (tener ben presente il
fatto che una volta raggiunto uno specchio, la luce si propaga come
un'onda emisferica).
\begin{figure}[htb]
  \begin{center}
    \begin{picture}(50,65)
      \put(10,10){\vector(1,3){15}}\put(10,10){\line(1,0){15}}
      \multiput(25,10)(0,10){5}{\line(0,1){5}}
      \put(10,55){\line(1,0){30}}\put(25,55){\vector(1,-3){15}}
      \put(35,5){\line(1,1){10}}
      \multiput(10,55)(5,0){7}{\line(1,3){3}} \put(30,50){\tiny
        Specchio}\put(42,8){\tiny Specchio
        semi-rifrangente}\put(17.5,5){$\mathrm{v}t$} \put(22,25){$l$}
      \put(13,36){$ct$}
    \end{picture}
    \caption{Percorso della luce in direzione verticale}
    \label{fig:mim1}
  \end{center}
\end{figure}
Dal disegno si capisce (\`e sufficiente usare il teorema di Pitagora)
che $(ct)^2=l^2+(\mathsf{v}t)^2$, da cui risulta:
\begin{equation}
  t_{\perp}=2t=\frac{2l}{c}\frac{1}{\sqrt{1-\frac{\mathsf{v}^2}{c^2}}}.
  \label{eq:mim1}
\end{equation}
L'altro raggio invece fa il percorso, nell'etere, in figura
\ref{fig:mim2}. \setlength{\unitlength}{1mm}
\begin{figure}[!h]
  \begin{center}
    \begin{picture}(65,28) \put(7.5,7.5){\line(1,1){10}}
      \put(6,18){\scriptsize Specchio
        semi-rifrangente}\put(12.5,12.5){\line(1,0){36}}
      \put(27.5,12.5){\vector(1,0){8}}
      \put(27.5,12.5){\vector(-1,0){5}} \put(48.5,2.5){\line(0,1){20}}
      \multiput(48.5,2.5)(0,4){6}{\line(1,1){5}}
      \put(12.5,12.5){$\underbrace{\makebox[36mm][c]{}}$}
      \put(29,7){$L$}\put(57,12.5){Specchio}
    \end{picture}
    \caption{Percorso della luce in direzione orizzontale}
    \label{fig:mim2}
  \end{center}
\end{figure}
Si pu\`o facilmente capire che
$t_{\sslash}=\frac{l}{c-v}+\frac{l}{v+c}$, da cui:
\begin{equation}
  t_{\sslash}=\frac{2l}{c}\cdot\frac{1}{1-\frac{\mathsf{v}^2}{c^2}}
  \label{eq:mim2}
\end{equation}
Dal momento che $ \mathsf{v} < < c $ possiamo sviluppare in serie di $v^2 / c^2$
la (\ref{eq:mim1}) e la (\ref{eq:mim2}), ottenendo:
\begin{equation}
  t_{\perp}=\frac{2l}{c}\left(1+\frac{1}{2}\frac{\mathsf{v}^2}{c^2}\right)
\end{equation}
\begin{equation}
  t_{\sslash}=\frac{2l}{c}\left(1+\frac{\mathsf{v}^2}{c^2}\right)
\end{equation}
In definitiva $t_{\sslash} -
t_{\perp}\simeq\frac{\mathsf{v}^2l}{c^3}$. Sempre facendo riferimento
alla figura \vref{fig:mm}, si vede che i raggi, dopo esser tornati
indietro, interferivano. Con un interferometro (il rivelatore della
figura), si poteva misurare quest'interferenza. Lo sfasamento
spaziale, in termini di lunghezza d'onda, \`e:

\begin{displaymath}
  \Delta\lambda=c(t_{\sslash}-t_{\perp})\simeq\frac{\mathsf{v}^2l}{c^2}
\end{displaymath}


Il numero di frange che si doveva dunque rilevare con l'interferometro
era:
\begin{displaymath}
  \frac{\Delta n}{n} = 
  \frac{\Delta\lambda}{\lambda}=\frac{l}{\lambda}\frac{\mathsf{v}^2}{c^2}
\end{displaymath}
L'esperimento del 1887 (quello eseguito con i migliori accorgimenti
per diminuire gli errori di misura), aveva $l=11m$,
$\lambda=6\cdot10^{-7}m$, $ \frac{v}{c}=10^{-4} $. Da questi dati
risulta $\Delta n / n =0.18 \pm 0.01$. Tuttavia i risultati
sperimentali davano $n = 0$: questo fatto rimaneva inspiegabile:
l'etere si muoveva con la terra? Come era possibile interpretare tale
fenomeno?
\begin{observazione}[Fitzgerald e Lorentz] Una prima risoluzione del
  problema fu avanzata da Fitzgerald (1889), il quale asseriva che il
  braccio parallelo alla direzione del moto, si contraeva di un
  fattore $\sqrt{1-\mathsf{v}^2/c^2}$. Poi venne Lorentz, che
  afferm\`o che, nel caso fosse vera la contrazione di Fitzgerald,
  allora anche i tempi del sistema in moto rispetto all'etere dovevano
  dilatarsi dello stesso fattore $\sqrt{1-\mathsf{v}^2/c^2}$. In tal
  modo egli trova le trasformazioni tra i sistemi di riferimento che
  portano il suo nome (e che andremo a ricavare dopo).
\end{observazione}
\section{Princ\`ipi di relativit\`a}
Successivamente la cosa venne affrontata da un punto di vista
matematico: tra il 1902 e il 1905, Poincar\`e si avvicin\`o al
principio di relativit\`a einsteniano, dicendo:
\begin{principio}[di Relativit\`a di Poincar\`e]\index{principio di
    relativit\`a!di Poincar\'e}
  Le leggi della fisica devono essere covarianti (avere cio\`e la
  stessa forma) per sistemi inerziali.
\end{principio}
Poincar\`e trov\`o dunque le trasformazioni di coordinate che resero
l'elettrodinamica compatibile con il principio di relativit\`a. Tali
trasformazioni sono le trasformazioni di Lorentz, che costituiscono un
gruppo; tale gruppo pu\`o essere ampliato ulteriormente, mantenendo la
sua struttura, dando origine al gruppo di Poincar\`e.

Venne poi la volta di Einstein (1905), il quale pose alla base della
sua teoria due principi:
\begin{dinglist}{192}
\item Principio di relativit\`a
\end{dinglist}
\begin{dinglist}{193}
\item La velocit\`a della luce \`e uguale in tutti i sistemi di
  riferimento inerziali
\end{dinglist}
Quale fu la grande innovazione di Einstein rispetto a Poincar\`e?  Il
fatto che egli elimin\`o il concetto di etere.\footnote{Ad inizio
  capitolo abbiamo imbrogliato il lettore, dicendo che la relativit\`a
  ristretta nasce per i contrasti tra elettromagnetismo e meccanica:
  Einstein, nello sviluppare la sua teoria, venne invece mosso dalla
  convinzione che non potesse esistere un sistema di riferimento
  privilegiato, ovvero di un moto assoluto, che portasse delle
  asimmetrie nell'elettromagnetismo; che questo risolvesse i contrasti
  di cui abbiamo parlato \`e un altro discorso, poich\'e non era
  possibile risolvere il dilemma di Einstein senza appianare i
  contrasti tra meccanica ed elettrodinamica.}


Successivamente, dall'invarianza di $c$, Einstein, con una serie di
esperimenti ideali, ricav\`o tutta una serie di leggi, tra cui le
trasformazioni di Lorentz. Cominciamo con lo scrivere gli assiomi alla
base della sua teoria.
\begin{dinglist}{202}
\item Omogeneit\`a e assolutezza dello spazio e del tempo, e isotropia
  (non vi sono differenze tra le direzioni) dello spazio. \footnote{In
    realt\`a questo primo assioma era implicito nella sua trattazione}
\end{dinglist}
\begin{dinglist}{203}
\item Si ha l'equivalente del principio di relativit\`a di Poincar\`e:
  \begin{principio}[di Relativit\`a einsteniano]\index{principio di
      relativit\`a!di Einstein} Le leggi della
    fisica hanno la stessa forma in tutti i sistemi inerziali.
  \end{principio}
\end{dinglist}
\begin{dinglist}{204}
\item La velocit\`a della luce nel vuoto\footnote{Ossia assenza di
    materia} \`e la stessa in tutti i sistemi di riferimento inerziali
\end{dinglist}
\begin{osservazione}
  Osserviamo che \ding{202} era valido anche prima della teoria della
  relativit\`a di Einstein
\end{osservazione}
\begin{osservazione}
  Nella fisica pre-relativistica \ding{203} era sostituito dal
  principio di \emph{relativit\`a galileiano}, il quale asseriva:
\end{osservazione}
\begin{principio}[di Relativit\`a di Galileo]\index{principio di
    relativit\`a!di Galileo} Le leggi della meccanica hanno la stessa
  forma in tutti i sistemi inerziali.\footnote{Come gi\`a notato,
    ci\`o era impossibile per l'elettromagnetismo, se si usavano le
    trasformazioni di Poincar\'e.}
\end{principio}
\begin{osservazione}
  In fisica pre-relativistica \ding{204} era sostituita
  dall'invarianza degli intervalli spaziali (\ref{eq:invspazio}) e
  temporali (\ref{eq:invtempo})
\end{osservazione}
