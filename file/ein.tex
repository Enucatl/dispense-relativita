%#!latex tesi.tex
\chapter{\ein \label{sez_ein}}
\minitoc
In questa sezione percorreremo una parte dell'articolo~\cite{alberto}
sulla relativit\`a, che tanto rese famoso \ein. Ne analizzeremo solo
le parti utili al nostro scopo.

Il punto di partenza di Einstein furono le asimmetrie presenti nella
teoria di Maxwell sotto le trasformazioni di Galileo: egli non si
capacitava del fatto che le equazioni non valessero pi\`u passando da un
sistema di riferimento inerziale ad un altro. Questo, assieme ai
risultati negativi dell'esperimento di Michelson e Morley, lo port\`o ad
asserire che era impossibile parlare di un sistema di riferimento
assoluto, e quindi di etere. Quindi enunci\`o il
%Infatti egli scrive
%\begin{citaz}
%  \`E risaputo che l'elettrodinamica di Maxwell, applicata a corpi in
%  movimento, porta ad asimmetrie che sembrano non centrare col
%  fenomeno.
%\end{citaz}
%\begin{citaz}
%  Esempi di questo genere, e l'impossibilit\`a di trovare il moto
%  della terra rispetto all'etere\footnote{Che egli chiama, in questo
%    contesto, ``light medium''.} suggeriscono che i
%  \underline{fenomeni elettrodinamici e meccanici} non possiedono
%  propriet\`a relative a sistemi di riferimento assoluto. Ci\`o porta
%  al
%\end{citaz}
\begin{principio}[di relativit\`a di \ein]
  Le stesse leggi dell'elettrodinamica e dell'ottica sono valide per
  tutti i sistemi di riferimento per i quali le equazioni della
  meccanica valgono.
\end{principio}
\begin{observazione}
  Il principio appena formulato sembrerebbe lasciar intendere che \ein{}
  ritenesse covarianti le sole leggi dell'elettromagnetismo, in quanto
  nel principio sono menzionate solo le leggi dell'elettromagnetismo e
  dell'ottica. Tuttavia non \`e cos\`i: egli vuole estendere il proprio
  principio a tutti i sistemi di riferimento inerziale, e assume
  implicitamente per buono il principio di relativit\`a galileiano (per
  il quale le leggi della meccanica hanno la stessa forma in tutti i
  sistemi di riferimento). Nell'ultima parte del suo articolo infatti fa
  sempre riferimento alla stessa legge della meccanica ($F = ma$), per
  entrambi i sistemi di riferimento che considera, cosa che non sarebbe
  vera se non assumesse che esse hanno la stessa forma. Se ci\`o non
  bastasse, il lettore potr\`a convincersi leggendo la
  sezione~\ref{relativita}.
\end{observazione}
%
%\begin{citaz}
%  [\ldots] Introdurremo poi un altro postulato, che \`e solo
%  apparentemente in contraddizione con il precedente, ossia che la la
%  luce si propaga nel vuoto con una velocit\`a definita, $c$,
%  indipendente dallo stato di moto del corpo che la emette.
%\end{citaz}

Successivamente \ein, nel suo articolo, pose la velocit\`a della luce
costante, e tratt\`o il concetto di simultaneit\`a e sincronizzazione:
chiariamo cosa intende \ein{} con tali termini. Sia dato un orologio nel
punto $A$, ed un orologio nel punto $B$, non in moto relativo tra loro,
e tali da scandire il tempo allo stesso modo. Essi si dicono sincroni
se, detto $\Delta t$ l'offset tra i due orologi, e detto $t_{A}$
l'istante in cui $A$ rileva l'avvenimento di un certo evento $\iota$,
allora $B$ rileva che un evento simultaneo a $\iota$, chiamamolo
$\iota'$, ha luogo all'istante $t_{A} + \Delta t$. Questo deve valere di
ogni evento $\iota$.

Ma cosa vuol dire che due eventi sono simultanei? Che essi sono
raggiunti da un raggio di luce emesso da una sorgente posta a met\`a
strada tra di loro, al medesimo istante. Tuttavia la definizione non \`e
per niente comoda, e conviene trovarne una pi\`u adatta per il concetto
di sincronizzazione, e, a detto di molti, \ein{} \emph{s'ispira} alla
definizione di Poincar\'e data in~\cite{carro3}.

Si prendano due osservatori, in $A$ e in $B$, solidali ad un certo
sistema di riferimento, e in corrispondenza ad essi si pongano due
orologi, uguali tra di loro. Si faccia partire da $A$ un raggio di luce,
al suo tempo $t_A$; tale raggio arrivi in $B$ al suo tempo $t_B$, e
ritorni in $A$ al tempo, di $A$, $t_A'$. I due orologi saranno sincroni se
%\begin{citaz}
%  Facciamo partire un raggio di luce, al ``tempo di $A$'' $t_A$, da
%  $A$ verso $B$. Al ``tempo di $B$'' $t_B$, esso \`e riflesso in $B$
%  verso $A$, dove \`e recepito al ``tempo di $A$'' $t'_A$. In accordo
%  con la definizione i due orologi sono sincronizzati se
%\end{citaz}
\begin{equation}
  t_{B} - t_{A} = t_{A}' - t_{B}.  
  \label{eq:sincroni}
\end{equation}
A questo punto \ein{} fa subito notare come orologi sincroni in un
sistema di riferimento, non lo siano pi\`u in un altro in movimento
rispetto al primo, dacch\'e la velocit\`a del raggio luminoso
dev'essere la stessa, per quanto da \ein{} postulato, in entrambi i
sistemi.

Nel seguito egli passa a ricavarsi le trasformazioni di Lorentz in un
modo non solo del tutto differente da quello da quello di Lorentz, e da
quello che potrebbe aver utilizzato \poin\footnote{Da tale dimostrazione
si sospetta fortemente che \ein{} avesse capito autonomamente questa
nuova teoria, anche perch\'e la dimostrazione \`e fortemente legata alla
costanza della velocit\`a della luce, al principio di relativit\`a, e al
concetto di simultaneit\`a.}, ma anche pi\`u mirato (\poin{} aveva
fissato un parametro, la velocit\`a della luce, per far tornare le
cose). Dunque i princ\`ipi che aveva postulato lo
portarono in maniera naturale alle trasformazioni\footnote{Non successe
alla maniera di Lorentz, il quale aveva trovato le trasformazioni pi\`u
per necessit\`a, ossia per spiegare i risultati dell'esperimento di
Michelson.}, senza dover aggiungere ipotesi \emph{ad hoc} per aggiustare
le cose a posteriori (com'era abituato a fare Lorentz in questo
campo). Ripercorriamo dunque la strada da \ein {} seguita.

\section{Le trasformazioni di Lorentz}
Prendiamo due sistemi di riferimento inerziali, $S$ ed $S'$ con
coordinate $(x,y,z,t)$ e $(\xi,\eta,\zeta,\tau)$, e con origini,
rispettivamente, $K$ e $\kappa$. Il sistema $S'$ si muove con velocit\`a
$v$ (verso destra, per concretizzare) rispetto al primo, in maniera
tale che gli assi delle $x$ e delle $\xi$ rimangano
paralleli. Vogliamo determinare le coordinate di $S'$ in funzione di
quelle di $S$.

A questo scopo osserviamo che un punto fisso di $S'$ si pu\`o mappare
in un punto fisso di $S$ con la posizione $x' = x - vt $. Cominciamo
col definire $\tau = \tau (x',y,z,t)$, e prendiamo due orologi
sincronizzati in $S'$, uno in $\kappa$, ovvero nell'origine, mappato in $0$
tramite $x-vt$, e l'altro tale che sia mappato in $x'$. La definizione
di sincronizzazione ci assicura che $\tau_{1} - \tau_{0} = \tau_{2} -
\tau_{1}$, che, una volta scritta $\tau$ come funzione, porge
(ricordiamo che la prima coordinata \`e tutta mappata)
\begin{equation}
  \frac{1}{2}\left[\tau(0,0,0,t) + \tau(0,0,0,t + \frac{x'}{c-v} + 
    \frac{x'}{c+v}) \right] =
  \tau(x',0,0,t + \frac{x'}{c-v});
\end{equation}
a questo punto, se $x'$ \`e scelto infinitesimo, possiamo scrivere,
dopo un paio di passaggi
\begin{equation}
  \frac{1}{2} \left( \frac{1}{c-v} + 
    \frac{1}{c+v} \right) \frac{\partial \tau}{\partial t}
  = \frac{\partial \tau}{\partial x'} + 
  \frac{1}{c-v}\frac{\partial \tau}{\partial t}
\end{equation}
o
\begin{equation}
  \frac{\partial \tau}{\partial x'} + 
  \frac{v}{c^{2} - v^{2}} \frac{\partial \tau}{\partial t} = 0.
\end{equation}
La medesima procedura con un orologio posto in $\kappa$ ed uno posto
sull'asse delle $y$, porge
\begin{equation}
  \frac{1}{2}\left[\tau(0,0,0,t) + \tau(0,0,0,t + 
    \frac{2y}{\sqrt{c^{2} - v^{2}}} \right] =
  \tau(0,y,0,t + \frac{y}{\sqrt{c^{2}-v^{2}}}),
\end{equation}
da cui $\partial \tau / \partial y = 0$. Analogamente per $z$. Dalla
linearit\`a di $\tau$
\begin{equation}
  \tau = a(v) (t - \frac{v}{c^{2} - v^{2}}x')
\end{equation}
dove $a(v)$ \`e una funzione per ora sconosciuta ma tale che $\tau =
0$ quando $t=0$.

Sapendo ora che un raggio di luce si propaga con la stessa velocit\`a
in tutti i sistemi, possiamo scriverne la legge oraria per la
coordinata lungo la quale si propaga. Per concretizzare $\xi = c
\tau$, se esso \`e emesso per $\tau = 0$, ossia
\begin{equation}
  \xi = ac (t - \frac{v}{c^{2} - v^{2}}x').
  \label{eq:xiti}
\end{equation}
E per il sistema $S$ tale raggio segue la legge $x = ct = x' + vt$,
che porta a
\begin{equation}
  t = \frac{x'}{c-v}.
\end{equation}
Sostituendo nella~\eqref{eq:xiti}, otteniamo
\begin{equation}
  \xi = a \frac{c^{2}}{c^{2} - v^{2}} x'.
  \label{eq:xi}
\end{equation}
In maniera del tutto analoga
\begin{equation}
  \eta = c \tau = ac (t - \frac{v}{\sqrt{c^{2} - v^{2}}} x' );
\end{equation}
quando $y / \sqrt{(c^{2} - v^{2})} = t$ e $x' = 0$ abbiamo
\begin{equation}
  \eta = a \frac{c}{\sqrt{c^{2} - v^{2}}} y
\end{equation}
e, senza ripetere tutto
\begin{equation}
  \zeta = a \frac{c}{\sqrt{c^{2} - v^{2}}} z.
\end{equation}
Se a questo punto sostituiamo $x'$ con $x - vt$, e poniamo $a (v)=
\varphi (v) \sqrt{1-v^{2}/c^{2}}$ , otteniamo
\begin{equation}
  \begin{array}{rcl}
    \tau & = & \varphi (v) \gamma (t - vx/c^{2})\\
    \xi   & = & \varphi (v) \gamma (x - vt)\\
    \eta & = & \varphi (v) y \\
    \zeta & = & \varphi (v) z		
  \end{array}
\end{equation}
dove $\gamma = 1/\sqrt{1 - v^{2}/c^{2}}$. A questo punto non manca che
determinare $\varphi (v)$. Con considerazioni analoghe a quelle di
\poin, anche \ein{} trova che $\varphi(v) = 1$. Non riportiamo la
dimostrazione per brevit\`a, e per non annoiare il lettore.

\section{Composizione delle velocit\`a ed equazioni di Maxwell}
Proseguendo \ein{} trova il teorema di composizione delle velocit\`a (la
cui forma \`e uguale a quella dell'equazione~\eqref{eq:vel}),
e dimostra, in maniera differente da quella di \poin, ma di scarso
interesse per noi, che con queste trasformazioni le equazioni di
Maxwell rispettano il principio di relativit\`a\footnote{Di scarso
  interesse perch\'e una volta trovate le trasformazioni, ci sono
  molti modi, tutti ugualmente corretti e non troppo difficili, di
  provare la covarianza delle equazioni.}. Ci risparmiamo dunque di
riportare questa parte di articolo.

\section{Significato fisico delle trasformazioni di Lorentz}
Tuttavia \ein, prima del teorema di composizione delle velocit\`a e
delle equazioni di Maxwell, scrive un paragrafo sul significato fisico
delle trasformazioni appena trovate, per spiegare come influiscono sui
corpi e sugli orologi in movimento. In questo paragrafo spiega come, in
conseguenza delle trasformazioni di Lorentz, la lunghezza di un corpo
rigido si contragga; fa notare che tale fenomeno \`e reciproco. Per
quanto riguarda i tempi, invece, \ein{} \`e meno chiaro, perch\'e non
esplicita la reciprocit\`a della dilatazione degli intervalli
temporali. Vediamo in dettaglio come procede.

Per prima cosa prendiamo una sfera, rigida\footnote{Approssimativamente
rigida: non esistono, infatti, in relativit\`a ristretta, corpi
rigidi.}, di raggio $R$, ferma nel sistema $S'$ (i due sistemi sono gli
stessi di prima). Il luogo dei suoi punti potr\`a scriversi come
\begin{equation}
  \xi^{2} + \eta^{2} + \zeta^{2} = R^{2},
\end{equation}
ovvero, in $S$
\begin{equation}
  x^{2} \gamma^{2} + y^{2} + z^{2} = R^{2};
\end{equation}
dunque in $S$ la sfera di $S'$ \`e vista come un ellissoide di assi 
$R / \gamma,\, R,\, R$: essa \`e dunque vista contratta; ma non solo
viene vista: essa \`e contratta a tutti gli effetti per il sistema
fermo, e la cosa \`e reciproca.

Per quanto riguarda i tempi prendiamo un orologio che in $S$
misurerebbe il tempo $t$, mentre in $S'$ misura il tempo $\tau$, e
domandiamoci che relazione intercorre tra i due tempi, una volta che
tale orologio \`e messo in moto assieme al sistema $S'$.

Dalle trasformazioni di Lorentz
\begin{equation}
  \tau = \gamma (t - v x /c^{2}) \stackrel{x = vt}{=} 
  \gamma t (1 - v^{2}/c^{2}) = t / \gamma 
\end{equation}
e quindi $\tau = t - (1 - \gamma^{-1}) t$: l'orologio perde $(1 -
\gamma^{-1})$ secondi ogni secondo di $S$.  Successivamente afferma che
se due orologi in $A$ e $B$, nel medesimo sistema di riferimento, sono
sincroni, e l'orologio in $A$ comincia a muoversi verso $B$, lungo la
congiungente, con velocit\`a $v$, una volta arrivato non ha pi\`u la
stessa sincronizzazione con l'orologio in $B$,\label{err_tempi} in
quanto \`e rimasto indietro di un tempo $t ( 1 - \gamma^{-1})$, se $t$
\`e il tempo impiegato per il tragitto. Tuttavia se vediamo questo
processo secondo il sistema di riferimento dell'orologio inizialmente in
$A$, \`e l'orologio in $B$ che si avvicina (ma $A$ non \`e
inerziale). Questo \`e il cosiddetto paradosso dei gemelli; in realt\`a
non possiamo applicare le trasformazioni di Lorentz poich\'e l'orologio
subisce un'accelerazione quando parte, e successivamente, per fermarsi,
subisce una decelerazione. Non \`e quindi per niente chiaro dal testo di
\ein{} se questo punto fosse stato da lui pienamente compreso.
