%#!latex tesi.tex
\chapter{Un tentativo di confronto\ldots}
\minitoc
In quest'ultima sezione cercheremo, come anticipato, di confrontare i
lavori dei due scienziati. Il loro obiettivo principale fu quello di
spiegare il fallimento dell'esperimento di Michelson e Morley e trovare
delle trasformazioni di coordinate tramite le quali le equazioni di
Maxwell risultassero covarianti; tale obiettivo fu da entrambi
raggiunto. Ma i due lavori presentano molto differenze. Vediamo quali.
\section[Il punto di partenza]{Il punto di partenza e le
  trasformazioni di Lorentz}

\paragraph{\ein} Egli parte dalle asimmetrie dell'elettromagnetismo e
postula, grazie anche al fallimento dell'esperimento di Michelson e
Morley, sia il principio di relativit\`a, sia l'invarianza della
velocit\`a della luce. Scrive infatti nel suo articolo
\begin{citaz}
  \`E risaputo che l'elettrodinamica di Maxwell, applicata a corpi in
  movimento, porta ad asimmetrie che sembrano non centrare col
  fenomeno.
\end{citaz}

In questo modo riesce a giungere, con pochi passi e senza ulteriori
ipotesi, alle trasformazioni di Lorentz.

\paragraph{\poin} Il francese invece parte dai risultati contrastanti
tra il fenomeno dell'aberrazione e l'esperimento di Michelson; infatti
leggiamo in~\cite{carro1}:
\begin{citaz}
  A prima vista sembra che l'aberrazione della luce indichi che la
  terra sia in moto assoluto rispetto all'etere. Ma tutti gli
  esperimenti, compreso quello di Michelson e Morley, sembrano non
  rilevare tale moto.
\end{citaz}
 Grazie al principio di relativit\`a e all'assunzione (debita)
dell'omogeneit\`a e isotropia dello spazio egli giunge, con un
procedimento del tutto generale, a delle trasformazioni tra sistemi di
coordinate inerziali che gli garantiscono 
% che tuttavia non sono n\'e quelle di Galileo, n\'e
%quelle di Lorentz. Per trovarle egli pone, come gi\`a ricordato, $K =
%c^2$, nella~\eqref{eq:di}, cosa che gli garantisce, tramite il teorema
%di trasformazione delle velocit\`a, la costanza della velocit\`a della
%luce (su cui per\`o non insiste), 
la covarianza delle equazioni di Maxwell.

\paragraph{}
I due giungono entrambi alle trasformazioni di Lorentz, in maniera
totalmente differente; \ein{} tuttavia riusc\`i a trovare quelle
corrette senza ipotesi a posteriori, come sola conseguenza della
semplice assunzione che $c = cost$; diversamente per
\poin: la velocit\`a della luce \`e presente nelle trasformazioni
solamente perch\'e in tal modo il fallimento dell'esperimento di
Michelson e Morley era spiegato, e le leggi dell'elettromagnetismo
risultavano covarianti.

\section{Il principio di relativit\`a} \label{relativita} Sulla
formulazione del principio di relativit\`a non c'\`e alcun dubbio:
entrambi lo enunciarono in maniera corretta; oltre
ai princ\`ipi, gi\`a enunciati, possiamo citare altri due brani dei due;
\ein{} scrive nel suo articolo
\begin{citaz}
  Esempi di questo genere [le asimmetrie dell'elettromagnetismo], e
  l'impossibilit\`a di trovare il moto della terra rispetto
  all'etere\footnote{Che egli chiama, in questo contesto, ``light
  medium''.} suggeriscono che i \emph{fenomeni elettrodinamici e
  meccanici} non possiedono propriet\`a relative a sistemi di
  riferimento assoluto.
\end{citaz}
\poin {} invece in~\cite{carro1} scrive:
\begin{citaz}
 Se \`e possibile imprimere ad un intero sistema una traslazione
 uniforme, senza modificare altro all'interno di esso, le equazioni
 dell'elettromagnetismo non risultano modificate tramite alcune
 trasformazioni, che noi chiameremo \emph{di Lorentz}; dunque due
 sistemi, uno immobile, uno in traslazione, divengono l'immagine esatta
 l'uno dell'altro. 
\end{citaz}

\section[Il concetto di tempo]{Il concetto di tempo: simultaneit\`a e
  sincronizzazione}

\poin{} gi\`a nel 1898, in~\cite{carro5}, aveva affrontato il problema
della simultaneit\`a di due eventi, affermando che due eventi sono
simultanei se avvengono nello stesso posto e allo stesso tempo. Dire
\emph{allo stesso posto} non \`e affatto cosa semplice, perch\'e la
definizione risulta stringente: il francese fa notare, in~\cite{carro4}
(riprendendo, tra l'altro, quanto scritto in~\cite{carro5})
\begin{citaz}
  Non solo noi non abbiamo alcuna intuzione diretta dell'uguaglianza
  di due durate, ma non abbiamo neppure quella di simultaneit\`a di
  due eventi che avvengano in posti diversi.
\end{citaz}
Se infatti i luoghi sono diversi, due
eventi sono simultanei se sono raggiunti contemporaneamente da dei raggi
di luce emessi simultaneamente da una sorgente posta a met\`a strada tra
loro. Questo \`e ci\`o che si pu\`o capire dalla sua seguente
affermazione
\begin{citaz}
 [\ldots] osservano un fenomeno astronomico [\ldots] e ammettono che
 tale fenomeno \`e avvenuto simultaneamente in tutte le parti del mondo
 [\ldots] con una piccola correzione [dovuta alla sfericit\`a della terra].
\end{citaz}
Questo modo di procedere si rivela scomodo tuttavia (ma al momento
\poin{} non sembra preoccuparsene, dal momento che scrive in una rivista
filosofica): sarebbero necessari degli orologi sincroni, che dunque
avessero la possibilit\`a di determinare la simultaneit\`a o meno di due
eventi. Come?
%\`E stato notato come la definizione di simultaneit\`a di \ein{} sia
%stata probabilmente presa in prestito da quella di \poin{} che
%%%%%%%%%%%%%%%%%%%%%%%%%%%%%%%%%%%%%%
%  Controllare la seguente citazione %
%                                    %
%                                    %
%%%%%%%%%%%%%%%%%%%%%%%%%%%%%%%%%%%%%%
In~\cite{carro3} scrive
\begin{citaz}
  Supponiamo che due osservatori posti in luoghi differenti regolino i
  propri orologi con l'ausilio di segnali luminosi: cercheranno di
  correggere questi segnali per il tempo di trasmissione, ignorando,
  tuttavia, il movimento di traslazione, credendo, di conseguenza, che
  i segnali abbiano, in entrambi i sensi, la stessa velocit\`a. Dunque
  incroceranno le osservazioni, inviando da $A$ e $B$ un segnale, e un
  altro da $B$ ad $A$. Il tempo locale sar\`a il tempo segnato dagli
  orologi in questa maniera regolati.
\end{citaz}
Quest'ultima citazione, raffrontata a quella di \ein{}, che scrive
\begin{citaz}
  Facciamo partire un raggio di luce, al ``tempo di $A$'' $t_A$, da
  $A$ verso $B$. Al ``tempo di $B$'' $t_B$, esso \`e riflesso in $B$
  verso $A$, dove \`e recepito al ``tempo di $A$'' $t'_A$\ldots
\end{citaz}
non pu\`o non far pensare ad una, non citata, ispirazione, anche
perch\'e \ein{} sembra seguire lo stesso ragionamento di \poin: prima
definisce simultanei gli eventi che accadono nello stesso tempo e nello
stesso luogo, e si dichiara insoddisfatto della definizione nel caso in
cui gli
eventi non abbiano luogo nel medesimo posto, come si legge
in~\cite{alberto}
\begin{citaz}
 Una tale definizione di tempo sar\`a soddisfacente solo se il fenomeno
 avr\`a luogo ove sono posti gli orologi [\ldots] ma non lo sar\`a pi\`u
 quando dovremo connettere il tempo di una serie di eventi che han luogo
 in posti differenti.
\end{citaz}
e successivamente comincia a definire gli orologi sincroni, come gi\`a
affrontato nella sezione~\ref{sez_ein}, e come citato pi\`u sopra.

%Tanto pi\`u che \ein, impiegato all'Ufficio Brevetti di Berna, nella
%sezione ``elettromagnetismo'', aveva il compito di leggere e
%riassumere le principali pubblicazioni del campo%
%\footnote{Facciamo notare che il 5 giugno 1905 \poin{} present\`o una
%  nota di quattro pagine sulla relativit\`a, nota che si riserv\`o di
%  espandere (cosa che fece con~\cite{carro1}); il 9 giugno fu editata,
%  e nella settimana seguente arriv\`o a Berna, appunto dove \ein{}
%  lavorava.}.  
Inoltre Carl Seelig e Maurice Solovine affermarono di
aver parlato e discusso sul lavoro di \poin{}~\cite{carro4} assieme con \ein,
all'\emph{Academie des Sciences}, bench\'e \ein{} si mostr\`o sempre
reticente nelle sue dichiarazioni in proposito.

% Tuttavia possiamo farci delle domande in merito: perch\'e \ein{} non
% si sarebbe dovuto ispirare a \poin{} anche per il principio di
% relativit\`a, differente tra i due per quel che riguarda la
% meccanica?  \newline Se infatti~\cite{carro3} l'aveva colpito, e da
% esso aveva preso spunto, perch\'e non aveva approfondito l'opinione
% di Poincar\`e, leggendo magari~\cite{carro4} (libro nel quale
% troviamo la formulazione forse pi\`u corretta e concisa del
% principio di relativit\`a) dacch\'e egli si apprestava a pubblicare
% un articolo cos\`i rivoluzionario, in una rivista prestigiosa come
% gli \emph{Annalen der Physik}? Non sarebbe valsa la pena di
% documentarsi un pi\`u approfonditamente, trovata una fonte come
% \poin?  \newline
%Per\`o perch\'e nessuno dei sostenitori di \poin{} ha mai trovato
%nulla sul cosiddetto riassunto di \ein? Eppure questo avrebbe provato
%senza alcun dubbio la non originalit\`a del suo lavoro\ldots

\section{La relativit\`a dei tempi e delle lunghezze}
\paragraph{\ein} Egli osserva come la contrazione delle lunghezze sia
reciproca, cosa importantissima per la relativit\`a; infatti leggiamo, a
questo proposito, in~\cite{alberto}
\begin{citaz}
  \`E chiaro che gli stessi risultati valgano anche per corpi fermi
  nel sistema di riferimento ``stazionario'', se visti da un sistema
  in moto uniforme.
\end{citaz}
Tuttavia non\`e altrettando chiaro per gli intervalli di tempo, come
gi\`a mostrato nella sezione~\ref{err_tempi} a
pag.~\pageref{err_tempi}. La sua ambiguit\`a segue dal fatto che non si cura
del processo di accelerazione, non chiarendo che cade il
principio di relativit\`a (ristretta).

\paragraph{\poin} Si rende conto
di come non si possa decretare che una misura di tempo sia pi\`u o meno
giusta di un'altra (a differenza dunque di \ein, che privilegia
l'orologio che ``sta fermo''). Infatti in~\cite{carro5} \poin{} scrive
\begin{citaz}
  [\ldots] non c'\`e una maniera di misurare il tempo che sia pi\`u
  vera di un'altra; quella generalmente adottata \`e solamente la
  pi\`u comoda.
\end{citaz}
Egli invece non parla di misure spaziali esplicitamente, ma avendo
dimostrato che le trasformazioni di Lorentz formano un gruppo, possiamo
immaginare che avesse capito la reciprocit\`a delle contrazioni.

\section{Una nuova dinamica}
\paragraph{\poin} 
%Nel suo lavoro si rende conto della necessit\`a di
%una nuova dinamica, e comincia parlando della gravitazione. In questa
%parte della memoria \poin{} trova che la forza di attrazione tra due
%corpi dipende dalla velocit\`a, ma sostiene che gli effetti, essendo
%piccoli, non possono essere rilevati da osservazioni astronomiche
%(successivamente svilupper\`a una teoria della gravitazione scalare, e
%non tensoriale, giungendo perci\`o ad una metrica
%errata). Successivamente, sempre nella stessa memoria~\cite{carro1},
%scrive
Trova il modo in cui trasformano le forze di origine elettromagnetica
(ricordato nelle equazioni~\eqref{eq:ftrax}-\eqref{eq:ftrayz}), e,
sempre in~\cite{carro1} si legge
\begin{citaz}
 [\ldots] tutte le forze, di qualsiasi origine siano, trasformano alla
 stessa maniera rispetto a quelle elettromagnetiche\ldots
\end{citaz}
Inoltre nella sezione dedicata alla gravitazione trova
l'equazione relativistica, corretta, della forza (vedi~\eqref{eq:forza_rel}).
\paragraph{\ein} %Se le equazioni dell'elettromagnetismo sono giuste, e
%le trasformazioni di Galileo errate (sotto le quali
%l'elettromagnetismo non \`e covariante), \`e necessario trovare una
%nuova dinamica, covariante sotto le trasformazioni di Lorentz. 
Sembra accorgersi solo in parte delle consegueza che una nuova
dinamica pu\`o avere:
% Gi\`a se ne era parlato nella sezione~\ref{relativita}: enunciando
% in maniera differente il principio di relativit\`a, \ein{} non si
% rende conto che \`e necessario fondare una nuova dinamica: infatti
% se le equazioni dell'elettromagnetismo erano giuste, e le
% trasformazioni di Galieo errate, al contrario di quelle di Lorentz,
% \`e necessario fondare una nuova dinamica, covariante sotto le
% trasformazioni trovate.  Invece nell'articolo non si trova traccia
% di questa ridefinizione: anzi, a fine articolo, \S{} 10, trattando
% della dinamica dell'elettrone debolmente accelerato, sbaglia
nel \S 10 del suo lavoro, scrive l'equazione di Newton in maniera non
corretta: da un sistema di riferimento all'altro trasforma le
coordinate spaziali e le componenti della forza, ma non la massa $m$
dipendente dalla velocit\`a (introdotta nella prima formulazione della
teoria della relativit\`a):
ottiene quindi dei risultati errati per massa trasversale e
longitudinale, come egli stesso ammetter\`a pi\`u avanti.

Questo \`e dovuto al fatto che scrive l'equazione della
dinamica non come \poin, bens\`i
\begin{displaymath}
 m \frac{\de \bef{x}^2}{\de t^2} = \varepsilon \bef{F} \mbox{ in } S,
 \mbox{ e } m \frac{\de \bef{x'}^2}{\de \tau^2} = \varepsilon \bef{F'}
 \mbox{ in } S', 
\end{displaymath} 
con l'errore di mantenere la stessa massa $m$ per le equazioni in sistemi
diversi. 
\section{Etere}
\paragraph{\ein} Gi\`a nella prima pagina del suo articolo scrive
\begin{citaz}
  L'introduzione di un ``etere luminifero'' \`e superflua, in quanto
  il punto di vista che sar\`a sviluppato in queste pagine non
  richieder\`a uno ``spazio assoluto e stazionario'', dotato di
  speciali propriet\`a.
\end{citaz}
Mette subito in chiaro il suo punto di vista: non \`e necessario
introdurre l'etere per parlare di fisica. Su questo, nell'articolo del
'05, \`e cos\`i chiaro che non lascia adito a dubbi.

\paragraph{\poin} Al contrario ci sono molteplici affermazioni di
\poin{} sull'etere; questo pu\`o essere spiegato poich\'e egli era, oltre
che uno dei pi\`u grandi matematici del secolo, sia
fisico sia filosofo (si tenga presente che articoli
come~\cite{carro5}, sono stati pubblicati in riviste filosofiche, e
non fisiche), e ad
alcune conclusioni giungeva come filosofo (la non necessariet\`a
dell'esistenza dell'etere), ad altre come fisico (come si spiegava la
vibrazione delle onde elettromagnetiche nel vuoto?). In~\cite{carro4}
egli scrive
\begin{citaz}
 Che l'etere esista realmente, \`e interesse dei
 metafisici [\ldots] verr\`a un giorno in cui l'etere sar\`a considerato
 come inutile [\ldots] Queste ipotesi non giocano che un ruolo
 secondario. Le possiamo sacrificare; non lo facciamo perch\'e
 l'esposizione perderebbe di chiarezza, ma questo \`e l'unico motivo.
\end{citaz}

Tuttavia in~\cite{carro1}, nel tentativo di riformulare la teoria
della gravit\`a, scrive
\begin{citaz}
  Se la gravit\`a si propaga con la velocit\`a della luce non \`e un
  caso: dev'essere cos\`i poich\'e \`e una funzione dell'etere; e
  cos\`i diventa necessario comprendere di che natura \`e questa
  funzione, e connetterla con le altre funzioni del fluido.
\end{citaz}
E ancora
\begin{citaz}
  [\ldots] per noi \`e essenziale che le cose vadano come se esistesse
  [l'etere], e che questa ipotesi [il principio di relativit\`a] \`e
  comoda per spiegare i fenomeni.
\end{citaz}

\paragraph{} A prima vista sembrerebbe che \poin{} non abbandoni
  l'etere; ma questo punto di vista \`e espresso parlando della
  gravitazione, della quale \ein{} non parla nel suo articolo. Affronta
  la questione, nell'ambito della teoria della relativit\`a generale in
  una conferenza del 5 maggio 1920, a Leiden, dicendo
\begin{citaz}
 Riassumendo, possiamo dire che per la relativit\`a generale lo spazio
 \`e dotato di propriet\`a fisiche; in questo senso potremmo dire che un
 etere esiste. In questa teoria uno spazio senza etere \`e
 inconcepibile, in quanto non solo la propagazione della luce sarebbe
 impossibile, ma non ci sarebbe alcuna possibilit\`a d'esistenza per i
 regoli e gli orologi, e, da ci\`o, per le distanze spazio-temporali, in
 senso fisico [\ldots ma] la nozione di movimento non deve essere ad
 esso applicata.
\end{citaz}


%questo, e
%l'aver pubblicato un articolo riassuntivo (al contrario del francese,
%che aveva esposto qua e l\`a molti punti importanti della
%relativit\`a) tutti i risultati fondamentali raggiunti, contribuirono
%all'attribuzione quasi unanime (e l'opinione dura ancora oggi) di
%tutta la teoria relativit\`a ristretta ad \ein, mentre, come abbiamo
%avuto modo di vedere, gi\`a prima di lui \poin{} aveva gettato le basi
%di questa nuova teoria.
